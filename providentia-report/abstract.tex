\begin{abstract}
    \edt{Large quantities of spatio-temporal data are captured everyday, whether by large web-based companies for social data or by other industries such as those concerning disaster relief or marine data analysis. This increases the need for back end systems to provide realtime query response times while scaling well (in terms of storage and performance) with increasing quantities of structured or semi-structured data.}
    
    \edt{Traditionally, relational database solutions have been used and, with technologies such as PostGIS, this solution has been well adapted in this regard. NoSQL databases have entered enterprises as the better suited data storage solution to scale with the increasing quantities of incoming data. However, the use of graph database technology (which falls within a subset of these technologies) has been rising in popularity and development. It is used as a means to handle structured and semi-structured data as a graph abstraction. Graph database technology has been found to handle graph-like data much more effectively when it comes to complex queries on interconnected data. The main problem posed by spatio-temporal data is that it has an inherent multi-dimensional property. This requires creative measures to effectively query, index, and store data with these properties.}
    
    \edt{This work is motivated by the need to effectively store multi-dimensional, interconnected data and to investigate whether or not graph database technology is better suited to address this when compared to the traditional relational database approach. The dataset used is the Yelp challenge dataset, which has an added complexity in that it holds various social properties, and the evaluation is based on how each database performs under data analysis scenarios similar to those found on an enterprise level. Three database technologies will be investigated using this dataset namely: PostgreSQL, JanusGraph, and TigerGraph. The dataset will be modelled and queried from each dataset, their query performance compared, and query languages within the scope of this problem will be discussed.}
\end{abstract}