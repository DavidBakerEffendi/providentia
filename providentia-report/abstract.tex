\begin{abstract}
    Large quantities of spatio-temporal data are captured everyday, whether by large web-based companies for social data or by other industries such as those concerning disaster relief or marine data analysis. This increases the need for backend systems to provide realtime query response times while scaling well (in terms of storage and performance) with increasing quantities of structured or semi-structured, multi-dimensional data. Traditionally, relational database solutions have been used and, with technologies such as PostGIS, this solution has been well adapted in this regard. However, the use of graph database technology has been rising in popularity and development and has been found to handle graph-like data much more effectively.
    
    This work is motivated by the need to effectively store multi-dimensional, interconnected data and to investigate whether or not graph database technology is better suited to address this when compared to the traditional relational database approach. Three database technologies will be investigated using this dataset namely: PostgreSQL, JanusGraph, and TigerGraph. The dataset used is the Yelp challenge dataset and the evaluation is based on how each database performs under data analysis scenarios similar to those found on an enterprise level.
\end{abstract}