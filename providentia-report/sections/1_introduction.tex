The use of graph database technology as well as the volumes of spatio-temporal data have increased in recent years \cite{mongovspostgres}. Graphs are composed of vertices (nodes) and edges (relations) joining two vertices together. In this paper, nodes are referred to as vertices and relations as edges. In a graph database, a vertex could be a user, business, vehicle, or city and edges could be a friendship, review, road, or location affiliation. Due to the relational nature of graphs, they can be used to model a variety of real-world phenomena such as physical, biological, social and information systems \cite{socialdata}.

Graph databases form part of a group of database technologies known as Not Only SQL (NoSQL) databases. This, and the aforementioned characteristics of graphs, make graph databases a flexible and scalable solution for many enterprise level problems. It is important to note that large streams of social media data are logged daily, a significant amount of which contain geotags and timestamps \cite{twitterdata}.

Although many categories of data can implicitly be interpreted in a graph structure, many traditional relational databases lack the architecture and high-level query languages to effectively model and manipulate this structure. Graph databases are designed to represent data as attributes on vertices and edges. They are often schema-less (no fixed data structure) and their attributes are described as key-value pairs. A major strength of graph databases are that they excel in complex join-style queries where relational databases often handle these inefficiently for large tables \cite{data-in-nosql}. In contrast, graph databases perform poorly in full graph aggregate queries (but fairly well when working in a local subgraph) whereas relational databases are efficient in this regard.

This research aims to clarify the suitability of which database technology is best suited to handle large-scale, spatio-temporal data. In particular, the investigation will cover the query speed, expressiveness of the available query languages, and computational demand on the open source graph database, JanusGraph, the open source object-relational database system, PostgreSQL, and the enterprise level graph analytics platform, TigerGraph.

\edt{The Yelp dataset is the dataset with which this investigation will be performed on. The Yelp dataset challenge is an annual challenge where students have a chance to conduct research using this dataset. The dataset is relevant to this research as it holds, among others, spatiotemporal properties. Hundreds of academic papers have been written using this dataset and winners of the challenge receive a prize based on their technical depth and rigor and other criteria as described in \cite{yelpdataset}.}

This will be achieved by; (i) modelling and importing the Yelp dataset into these three databases, (ii) benchmarking them against one another using various analysis that represent real-world contexts to fairly represent the type of querying and data manipulation the dataset would typically undergo, and (iii) comparing the suitability of JanusGraph with that of TigerGraph, by investigating their graph query language implementations.
\comB{Add a sentence to explain what the Yelp dataset is and that it is often used in yearly competitions.}