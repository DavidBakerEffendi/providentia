The research in this paper compares the problem of modelling, visualizing, and querying large-scale spatio-temporal data. Query performance, storage and computation cost, and simulated real-world application will be investigated. The Yelp dataset contains not only spatio-temporal properties but also social and linguistic aspects.

One of the core issues is how to store spatio-temporal data efficiently. Coordinates are comprised of latitude and longitude and time adds a third dimension. The accessing times of secondary storage is an issue when housing large volumes of data. The use B-Trees, R-Trees, and using a geo-graph\footnote{A grid-based geospatial system where vertices are grids and edges connect other vertices to these grid vertices to indicate their physical location.} are three techniques investigated for indexing uni- and multidimensional data efficiently. Briefly discussed is the level of difficulty while learning and writing three popular graph querying languages namely; Gremlin, Cypher, and GSQL.

Due to the rise of popularity in web-based applications, the ease of implementation of these three databases will be investigated with a Flask \cite{flask} driven back end and Angular \cite{angular} driven front end. This will show appropriate real-world application for graph database technologies in a production setting and any challenges during implementation versus a classical SQL database back end.
