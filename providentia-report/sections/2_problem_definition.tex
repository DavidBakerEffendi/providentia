The research in this paper compares the problem of modelling, visualizing, and querying large-scale spatio-temporal data \edt{using traditional relational database approaches versus a more modern graph database one}. \comB{Previous sentence confusing: Is ``comparing'' referring to the fact that various database options are compared w.r.t. these aspects.} Query performance, storage and computational cost, and ease and efficiency of simulation of real-world applications will be investigated. The Yelp dataset contains not only spatio-temporal properties, but also social and linguistic aspects.

One of the core issues is how to store spatio-temporal data efficiently. Coordinates are comprised of latitude and longitude and time adds a third dimension. The accessing times of secondary storage is an issue when housing large volumes of data. The use of B-Trees, R-Trees, and a geo-graph\footnote{A grid-based geospatial system where vertices are grids and edges connect other vertices to these grid vertices to indicate their physical location.} are three techniques investigated for indexing uni- and multidimensional data efficiently.

\edt{Effectively querying a graph topology is another problem which one needs to deal with. Basic graph pattern matching is a popular technique used to extract data from graph databases but there are a range of graph query languages which implement this and other, more complex, graph patterns. Section \ref{sec:graph-lang} addresses three graph querying languages namely; Gremlin, Cypher, and GSQL. The solution to how traditional relational operators such as union and difference are implemented in graph pattern matching are also covered under the aforementioned section. The conciseness of querying graphs versus traditional SQL is a result which will be addressed in the conclusion (Section \ref{sec:conclusion}). To effectively measure the difficulty of learning and writing these queries, one requires experiments involving developers new to these languages and measuring their progress, but this is out of the scope of this investigation.}

% Briefly discussed is the level of difficulty in learning and writing queries using these three popular graph querying languages namely; Gremlin, Cypher, and GSQL.

\comB{The problematic aspect w.r.t. the previous sentence is that in order to really do this, you need a few developers new to Gremlin, Cypher, and GSQL, and then you have to measure how long it takes them to write these queries and how often they make mistakes.}

Due to the rise of popularity in web-based applications, the ease of incorporating these three databases will be investigated with a Flask \cite{flask} driven back-end and Angular \cite{angular} driven front-end. This will show appropriate real-world application for graph database technologies in a production setting and any challenges during implementation versus a classical SQL database back-end.
