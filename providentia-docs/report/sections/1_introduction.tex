\tilo{The collection of spatio-temporal data is commonplace today: prime examples are user-generated data from mobile devices or social platforms, streams of sensor data from static or moving sensors, satellite or remote sensing data. To make sense of this vast volume of heterogeneous data, a variety of aggregation and fusion steps have to be applied to achieve semantic data enrichment for subsequent use in value-added services. However, enabling these enrichment steps generally relies on stable and highly scalable data handling and storage infrastructures. However, with ever increasing volumes of spatio-temporal data relational database technology (even with special object-relational extensions for spatio-temporal data) is stretched to its limits and currently new paradigms commonly referred to as 'Not Only SQL' (NoSQL) databases are tried out as more scalable replacements, see. e.g., \cite{mongoVsPostgres}.}

\tilo{Given that a significant amount of data logged daily with geotags and timestamps (see e.g., \cite{twitterData}) and considering the distributed and global/local nature of such spatio-temporal data, simple \textit{graph structures as basic representations} come quite natural: traffic information in a road network, the spreading of epidemics over adjacent geographic regions, or time series data for chains of events. In a nutshell, graphs are composed of vertices (nodes) and edges (relations) joining two vertices together. Graph databases form an integral part of a group of NoSQL database technologies offering flexible and scalable solution for many enterprise level problems. In a graph database, a vertex could be a user, business, vehicle, or city and edges could represent adjacency, roads, or location affiliations. Due to the relational nature of graphs, they are already used to model a wide variety of real-world phenomena such as physical, biological,and social information systems \cite{socialData}.}

\tilo{However,} although many categories of data are interpreted as a graph structures \tilo{today}, traditional relational databases \tilo{still} lack the architecture and high-level query languages to effectively model and manipulate \tilo{such structures}. Graph databases are designed to represent data as attributes on vertices and edges. They are often schema-less (\tilo{i.e. there is} no fixed data structure) and their attributes are described as key-value pairs. \tilo{Thus,} a major strength of graph databases \tilo{is} that they excel in complex join-style queries where relational databases \tilo{are notoriously} inefficiently for large tables \cite{dataInNosql}. In contrast, graph databases \tilo{tend to} perform poorly \tilo{when moving from local subgraphs to full graph aggregate queries, where relational database technology plays out its strength}.

This \tilo{paper investigates} the suitability of which database technology is best suited to handle large-scale, spatio-temporal data \tilo{in \textit{typical data manipulation tasks under real world settings}}. In particular, the investigation will cover the query speed, expressiveness of the available query languages, and computational demand on the open source graph database, JanusGraph, the open source object-relational database system, PostgreSQL, and the enterprise level graph analytics platform, TigerGraph. \edt{Two graph database technologies from two different implementations of property graph databases have been selected to measure the suitability of each implementation.}

\edt{We have selected two datasets with different representations of spatio-temporal properties, complexities from additional attributes, and total size.}

\tilo{To allow for a fair and authoritative evaluation we use a real world data set for benchmarking. Yelp is an internationally operating business directory service and review forum. The `Yelp dataset' is a current subset of Yelp's businesses, reviews, and user data comprising spatio-temporal data for 10 metropolitan areas across two countries with about 200,000 businesses and 6.6 million reviews. It is offered for academic use in the yearly yelp challenge \cite{yelpDataset}, for a variety of data analysis and discovery tasks.}

\edt{As an additional dataset to further strengthen our results, the ambulance response simulation dataset from Ume\r{a} University\footnote{From here forth will be referred to as the ``medical response dataset''.} will be used for our investigation. The medical response dataset holds the simulation results where, given a number of hospital resources such as dispatch centres and ambulances, emergency calls are responded to. The dataset records spatio-temporal properties such as the time intervals within the response life cycle and the origin and destination of the resource during the response. Other technical properties regarding the emergency itself are also recorded. The dataset only contains ambulance responses.}

Our \tilo{contributions can be summarized as follows}: (i) modelling and importing \tilo{a real world} \edt{and artificial} \tilo{dataset into three state of the art} databases, (ii) \tilo{rigorously} benchmarking \tilo{them} using various analyses that fairly represent the type of querying and data manipulation \edt{each} dataset would typically undergo, and (iii) comparing the suitability of JanusGraph \tilo{vs.} TigerGraph by investigating their graph query language implementations.