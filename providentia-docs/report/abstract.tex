\begin{abstract}
    \tilo{Every day} large quantities of spatio-temporal data are captured, whether by \tilo{Web}-based companies for social data \tilo{mining} or by other industries \tilo{for a variety of applications ranging from} disaster relief 
    tilo{to} marine data analysis. \tilo{Making sense of all this data dramatically increases the need for intelligent} backend systems to provide realtime query response times while scaling well (in terms of storage and performance) with increasing quantities of structured or semi-structured, multi-dimensional data. \tilo{Currently}, relational database solutions \tilo{with spatial extentions} such as PostGIS, \tilo{seem to come to their limits}. However, the use of graph database technology has been rising in popularity and has been found to handle graph-like data much more effectively.
    
    This work is motivated by the need to effectively store multi-dimensional, interconnected data and to investigate whether or not graph database technology is better suited to address this when compared to the traditional relational database approach. Three database technologies will be investigated using this dataset namely: PostgreSQL, JanusGraph, and TigerGraph. \edt{The datasets used are the Yelp challenge dataset and an ambulance response simulation dataset from Ume\r{a} University, Sweden}. The evaluation is based on how each database performs under data analysis scenarios similar to those found on an enterprise level.
\end{abstract}